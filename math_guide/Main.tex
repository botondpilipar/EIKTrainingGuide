% !TeX root = ./Main.tex

\documentclass[12pt]{article}

%packages
\usepackage{subfiles}
\usepackage[utf8]{inputenc}
\usepackage[hungarian]{babel}
\usepackage{t1enc}
\usepackage{amsmath}
\usepackage{graphicx}
\usepackage{enumitem}
\usepackage{wrapfig}
\usepackage{array}
\usepackage{multicol}
\usepackage{libertine}
\usepackage{geometry}

\renewcommand{\arraystretch}{1.6}
\setlistdepth{9}
 \geometry{
 a4paper,
 total={170mm,257mm},
 left=15mm,
 right=15mm,
 top=20mm,
 bottom=20mm
 }
 \graphicspath{{./images/}}
 \setlength{\parindent}{0em}
 \setlength{\parskip}{1em}

\title{Tanulásmódszertan tréning ősz}
\author{Pilipár Botond}
\author{Bagyura Gábor}
\author{Sebestény Veronika}
\author{Takács Eszter}
\date{\today}

\begin{document}
\subfile{chapters/0_frontpage}

\tableofcontents
\clearpage
\section{Összefoglalás}
\subfile{chapters/1_guide_content}

\section{Bemutatkozás}
\subfile{chapters/2_introduction}

\section{Miért tanulunk matekot?}
\subfile{chapters/3_motivation}

\section{Hatékony tanulás}
\subfile{chapters/4_methods}

\section{Általános tanácsok, tippek}
\subfile{chapters/5_tips_tricks_qa}

\end{document}
