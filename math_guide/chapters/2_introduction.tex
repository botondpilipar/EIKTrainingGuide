\documentclass[../Main.tex]{subfiles}

\begin{document}
\begin{center}
    \begin{tabular}{| m{1.3 em} | m{0.7\textwidth} | m{0.1\textwidth} |}
    \hline
    1. & EIK bemutatása, tréner bemutatkozása & 1 perc \\
    \hline
    2. & Ráhangoló kérdések & 2 perc \\
    \hline
    3. & Tréning témájának bemutatása & 2 perc \\
    \hline
    - & \textbf{A feladat teljes időtartama} & 5 perc \\
    \hline
    \end{tabular}
\end{center}

\subsection{Tréner bemutatkozáas}

Minden tréner annyit mondjon magáról, amennyit szeretne. Például: név, szakirány, EIK tagság,
egyedi hobbi, érdeklődés. A lényeg, hogy megteremtsük a befogadó légkört. Nem mint hivatalos
tréner szerepben szeretnénk mutatkozni, sokkal inkább mint egy átlag hallgató, aki ugyanazokon
a nehézségen átesett mint ők. \par

\subsection{Ráhangoló kérdések}
\begin{itemize}
    \item Kinek hogyan tetszik a matematikai alapismeretek? Könnyűnek vagy nehéznek érzitek?
    
    \textit{Ha kollektívan nehéznek érzik a tárgyat megnyugtathatjuk őket, hogy az első egyetemi félév
    egy hirtelen váltás a matematikában, és azért jöttünk, hogy tanácsokkal, módszerekkel segítsünk nekik.}

    \item Hogyan látjátok, miben más az egyetemi matematikai oktatás mint a középiskolai?
    
    \textit{Ezzel a kérdéssel arra szeretnénk felhívni a figyelmet, hogy az egyetemen folyamatosan
    készülni kell, mert folyamatos számonkérések vannak, és a matematika anyag megértésére, feldolgozására
    is több időt kell szakítani.}

    \item Az egyetemi órákra való felkészülésből, kinek hány százalék megy matalapra?
    
    \textit{Itt rá szeretnénk világítani arra, hogy a matematikai alapismeretek az első félév legintenzívebb
    és egyben legfontosabb tárgya. Megnyugtathatjuk őket, hogy ez teljesen természetes, illetve bátorítsuk őket,
    hogy hiába nagyon sok idő a felkészülés és lehet nem jön egyből eredmény, a matalap az első félév legfontosabb
    tárgya. Minden befektetett energia megtérül későbbi félévekben.}
\end{itemize}

\subsection{Tréning témájának bemutatása}

\textit{Pár mondatban vázoljuk a tréning nagyobb témáit, és tegyük hozzá, hogy ezek számos korábban (általuk)
felsorolt nehézségben segítséget tudnak adni.}

\begin{multicols}{2}
\begin{enumerate}
    \item Motiváció a matematika tanuláshoz
    \item Hatékony tanulás
    \begin{itemize}
        \item Tanulás fázisai
        \item Cornell módszer
        \item Gondolattérkép
        \item Egyszerű és összetett tanulási technikák
    \end{itemize}

    \columnbreak

    \item Általános tippek
    \begin{itemize}
        \item Időbeosztás
        \item Feladattípusok gyakorlása
        \item ZH és vizsga megírása
    \end{itemize}
\end{enumerate}
\end{multicols}
\end{document}