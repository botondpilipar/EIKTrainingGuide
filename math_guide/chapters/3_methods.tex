\documentclass[../Main.tex]{subfiles}
% \usepackage{enumit}

\begin{document}

\subsection{Bevezetés}

\begin{flushleft}
    Ebben a részben arról lesz szó, hogy mi a hatékony tanulás, milyen fázisokból áll egy tanulási folyamat,
    melyik fázis miért fontos és elengedhetetlen a folyamat során.
\end{flushleft}


\textit{\textbf{Tipp:} Érdemes először a kérdéseket feltenni a gólyáknak, majd hagyni, hogy agyaljanak rajta és utána közösen megbeszélni (célszerű rávezetni őket a válaszra, akkor jobban elmélyül az új ismeret).
}

\subsubsection{Miért akar az ember hatékonyan tanulni? \newline Miért fontos a hatékony tanulás?}

\begin{flushleft}
    Biztos te is hallottad már azt a mondást, hogy a halálod napjáig tanulsz.
    Na már most, ha az egész életünk arról szól, hogy tanulunk, akkor lehet nem ártana,
    ha ezt jól és hatékonyan tudnánk csinálni.
\end{flushleft}

    \textit{\textbf{Rávezető kérdések:} Meddig tanul az ember? Van e vége a tanulásnak az ember életében?}

\begin{flushleft}
    Fontos, mert ha jól csináljuk, akkor időt tudunk spórolni vele, tovább emlékszünk rá, többek leszünk általa
    rávezető kérdések: Mi lesz az eredménye annak, ha hatékonyan tanulunk? Mit kapunk a hatékony tanulás által?
\end{flushleft}
    
\subsubsection{Mit is jelent az, hogy hatékony tanulás?}

\begin{flushleft}
    Azt jelenti, hogy valamit képesek vagyunk olyan szinten elsajátítani, hogy egyrészt hosszú távon megmarad,
    tehát nem felületes a tudásunk, másrészt tudunk rá építeni, tehát értjük is,
    amit megtanultunk és harmadrészt pedig képesek vagyunk felismerni és felhasználni akár a mindennapi életünkben,
    akár a munkánk, hivatásunk során.
\end{flushleft}

\textit{\textbf{Rávezető kérdések:} Meddig lenne jó emlékeznünk a megszerzett tudásra? Miért szükséges számunkra az új tudás, mit akarunk vele kezdeni?}
\end{document}

\subsubsection{Hogyan is kell ezt jól csinálni?}

\textit{\textbf{Tipp:} itt elég egy minimális választ megvárni, a kitalálhatóbbakat pl.
 (elegendő idő, elszántság, türelem, kitartás, életünk szerves részévé tevés)
 és utána a többit érdemes hozzáfűzni, mert azok az új ismeretek}

 \begin{flushleft}
    Ahhoz, hogy ezt az úgynevezett hatékony tanulást el tudjuk sajátítani, az szükséges hozzá, hogy elszántak,
    türelmesek és kitartóak legyünk, biztosítsunk elegendő időt rá, ismerjük a tanulás 5 fázisát,
    amire felbomlik, ismerjük mindezekhez a megfelelő tanulási technikákat és ezeket mind az 
    életünk szerves részévé tegyük.
 \end{flushleft}

\textit{\textit{Rávezető kérdések}: Mi szükséges minden jó dologhoz?
 Mit kell tenni ha valami nem sikerül?
 Miből tevődhet össze a tanulás folyamata?}

 \subsubsection{De mi is ez az 5 fázis? \newline Melyik mit takar? \newline Miért fontos része a tanulási folyamatnak?}

\begin{enumerate}
    \item {Tervezés}
    \begin{enumerate}
        \item {Mit csinálsz ekkor?}
        \begin{enumerate}[label=i]
            \item legelső lépés legyen mindig az, hogy megtervezed a tanulási folyamatodat
            \begin{enumerate}
                \item Melyik tantárgy?
                \item Melyik anyagrész?
                \item Melyik nap?
				\item Mennyi időd van rá?
				\item Meddig akarsz eljutni vele aznap?
            \end{enumerate}
        \end{enumerate}
        \item Miért fontos ez?
        \begin{enumerate}
            \item mert így be tudod iktatni a hétről, hétre tanulást, ami a sikeres és hatékony tanulás egyik kulcsa
            \item mert így tisztábban fogod látni, hogy mikor meddig kell eljutnod ahhoz, hogy megtudd tanulni az anyagot
                 időre, ami segít a koncentrációdon is, mert akkor tényleg csak arra tudsz fókuszálni, ami aznapra van
        \end{enumerate}
    \end{enumerate}

    \item Előkészület
    \begin{enumerate}  
        \item Mit csinálsz ekkor?
        \begin{enumerate}
            \item elhelyezed magadat az adott témában
            \begin{enumerate}
                \item áttekinted a teljes nagy anyagrészt
                \item elolvasod az aznapra kijelölt anyagrészt
            \end{enumerate}
        \end{enumerate}
        \item Miért fontos ez?
        \begin{enumerate}
            \item mert ezzel kicsit belerázódsz az anyagba, ráhangolódsz
			\item mert ezzel már egy kicsit megismerkedsz a megtanulandó résszel
			
    \end{enumerate}
    \item Megértés
    \begin{enumerate}
        \item Mit csinálsz ekkor?
        \begin{enumerate}
            \item megpróbálod megérteni az anyagrészt
            \begin{enumerate}
                \item han em sikerül magadtól, akkor KÉRJ segítséget!!!
                \begin{enumerate}
                    \item tanártól
					\item szaktárstól
					\item Internet
					\item könyvtár
					\item EIK-esek (eotvos.informatikai.kor@gmail.com)
                \end{enumerate}
            \end{enumerate}
            \item megpróbálod összekapcsolni korábbi ismeretekkel, hogy tudd hova kötni
            \begin{enumerate}
                \item korábbiak elolvasása
			    \item kapcsolatok megkeresése
            \end{enumerate}
        \end{enumerate}
        \item Miért fontos ez?
        {\begin{enumerate}
            \item mert ha értjük és már tudjuk hova kötni az új tudást, akkor könnyebben tudjuk megtanulni
			\item mert így könnyebben tudjuk felhasználni ott, ahol szükséges
        \end{enumerate}}
    \end{enumerate}

    \item Megtanulás
    \begin{enumerate}
        \item Mit csinálsz ekkor?
        \begin{enumerate}
            \item elsajátítod annyira az anyagot, hogy segédanyag nélkül is TUDOD
		    \item TUDOD = fel tudod pontosan idézni, érted, tudod alkalmazni, tudsz hozzá kapcsolni új dolgokat
		    \item ha ezek közül akár csak egy nem teljesül, akkor még nem tanultad meg!
        \end{enumerate}
        \item Miért fontos ez?
        \begin{enumerate}
            \item mert így rögzül az agyadban hosszútávra
        \end{enumerate}
    \end{enumerate}

    \item Visszaidézés
    \begin{enumerate}
        \item Mit csinálsz ekkor?
        \begin{enumerate}
            \item megpróbálod visszamondani a tanultakat
        \end{enumerate}
        \item Miért fontos ez?
        \begin{enumerate}
            \item mert ezzel megtanuljuk előhívni a megszerzett tudást
            \begin{enumerate}
                \item egy dolog bejuttatni az agyba az információt és megint másik dolog azt az információt előhívni
                \begin{enumerate}
                    \item más idegi pályákon történik a bejuttatás és az előhívás, ezért fontos mindkettőt gyakorolni
                    \begin{enumerate}
                        \item amíg még friss a bejuttatott információ addig könnyebben tudunk rá emlékezni  így hamar ki
                         tudjuk építeni azt az idegi pályát, amin majd a későbbiekben is elő tudjuk hívni az új ismeretet
                        \item gyakorlással mélyítjük el az idegi pályákat
                        \begin{enumerate}
                            \item minél mélyebb, többször bejáratot az idegi pálya annál tartósabban, hosszabb ideig megmarad az információ
                        \end{enumerate}
                    \end{enumerate}
                \end{enumerate}
            \end{enumerate}
        \end{enumerate}
    \end{enumerate}
\end{enumerate}

