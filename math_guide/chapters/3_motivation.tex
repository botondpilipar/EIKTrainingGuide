\documentclass[../Main.tex]{subfiles}

\begin{document}

\begin{center}
    \begin{tabular}{| m{1.3 em} | m{0.7\textwidth} | m{0.1\textwidth} |}
    \hline
    1. & Matematika egyetemi továbbtanuláshoz & 5 perc \\
    \hline
    2. & Választott példa bemutatása & 5 perc \\
    \hline
    3. & Személyes élmény, tapasztalat ha van & 5 perc \\
    \hline
    - & \textbf{A feladat teljes időtartama} & 15 perc \\
    \hline
    \end{tabular}
\end{center}

\subsection{Bevezetés}

\begin{itemize}
    \item[--] Hogyan lehetséges, hogy az emberek egy része könnyen tanulja a matematikát, míg
mások robotolva, nagy erőfeszítések árán képesek erre?
    \item[--] Miért van az, hogy a matek tanulás egyik pillanatban elkápráztatja az embert, a
másikban pedig kudarc élménnyé válik?
    \item[--] Miért fordul az elő, hogy egyes hallgatók középiskolában szerették a matematikát, az
egyetemi matematika viszont szenvedéssé vált számukra?
\end{itemize}

\begin{flushleft}

\textbf{Miért tanulunk matekot?} Talán nincs is olyan kérdés, amit ennél többször teszünk fel magunkban,
egy több órás analízis tanulás után. Még ha hallunk
is példákat gyakorlatokon, sokszor nem érezzük relevánsnak számunkra,
hiszen nem foglalkoztunk mégsem titkosítással, sem fizikai modellek megalkotásával,
ahol ez a fajta tudás elengedhetetlen. A következő pár oldalban szeretnénk nektek 
egy pár kézzel fogható példát adni, hol hatalmas segítség a matematika. 

\end{flushleft}

\subsection{Egyetem és matematika}

\begin{flushleft}
A matematikai tárgyak tanulása nyitja meg a kaput a mélyebb ismeretek felé, a tudományos munka felé.
BSc képzésen a szakirányok segítik a hallgatót döntésében, hogy milyen irányba szeretnének menni.
Mindazoknak, akik szeretnék tanulmányaikat folytatni MSc-n, illetve akár PHD-n,
erős matematikai alapokat kell biztosítani, tekintve, hogy a magasabb szintű problémák esetén ritkán kerülhető el,
hogy széleskörű matematikai tudásanyagot kelljen felhasználni. Bsc-n a hallgatók az alábbi 
matematikai területeket fogják megismerni:

\begin{itemize}
    \item Analizis
    \item Diszkrét matematika
    \item Numerikus módszerek
    \item Lineáris algebra
    \item Statisztika
\end{itemize}

\end{flushleft}

\begin{flushleft}
    Egyetemi szempontból az egyik fő probléma okozója a matematika tanulása.
    Mivel vannak olyanok, akik ha nehézségekbe ütköznek, tovább tudnak lépni,
    mások pedig a nehézség miatt abbahagyják a képzésüket már az elején, vagy akár a későbbiekben.
    Nehézésgek egyik fő okozója a rossz tanulási módszerek használata.
    Általában arra törekednek az emberek, hogy az elsajátítandó ismereteket többszöri ismétlés
    révén rögzítsék ("magolás"). Ennél léteznek jóval hatékonyabb módszerek, amikről később fogtok hallani.
\end{flushleft}

\subsection{Példák}

\begin{flushleft}
A matematika a rövidítések, csalások tudománya.
A legtöbb matematikai tétel a gyakorlatba átültetve megkönnyíti az életünket.
A matekos problémamegoldó gondolkodás nem csak papíron könnyíti meg a dolgunkat.
Ha hatékony és gyors programokat szeretnénk írni, használhatunk matematikailag bizonyított módszereket, formulákat.
\end{flushleft}

\begin{itemize}
    \item[--]  \textit{\textbf{{Választható példa: Fibonacci rekurzívan}}}
    
    Tegyüg fel, hogy szeretnénk Fibonacci számot kiszámoló programot írni, méghozzá minél hatékonyabban.
    Fibonacci számolása a rekurzió tipikus példája, melyben a függvény saját magát hívja meg. Ez a
    megközelítés azonban pazarló memória és számítási kapacitás tekintetében. Számos módon
    javítható a megoldás, ám a legalkalmasabb erre, ha a rekurzó helyett a Binet-formulát használjuk.

    \[ 
        f_n = \frac
            {
            (\frac{1 + \sqrt{5}}{2})^n - (\frac{1 - \sqrt{5}}{2})^n
            }
            {
                \sqrt{5}
            }
    \]

    Bár ez első pár szám esetében a formulás változat lassabb, mint a rekurzív változat,
    nagyobb számokra (pl. 35. fibonacci szám) a formulás változat nagyságrendekkel jobb.
    Ha formulával számolunk, a szám nagysága sokkal kevésbé befolyásolja a számítási igényt,
    mint a rekurzív képletnél. Azonban, ha az így kapott programunkat lefuttatjuk Fibonacci 72-re,
    azt látjuk, hogy egy eltérés van a valós eredmény és az általunk kapott eredmény között,
    vajon ez miért lehet? Hol csúszhatott el a számítás? Igazából, ez egy meglehetősen komplex kérdés,
    egy teljes tudományág foglalkozik ezzel (de nem csak ezzel): a numerikus módszerek.
    A válasz a számítógép véges lebegőpontos aritmetikájában rejlik.

    \item[--]  \textit{\textbf{Választható példa: Bloom filter algoritmus}}
     
        A matematikai módszerek az algoritmusok és adatszerkezetek területén is hatalmas jelentőséggel
        bírnak. Vegyük például a Bloom filter-t. Ez az elemi adatszerkezet a jelenleg ismert leggyorsabb
        módja, hogy egy elemről meghatározzuk, benne van-e egy nagy elemszámú halmazban. Az
        algoritmus gyorsaságának kulcsa hash függvényekben rejlik. A meghatározás algoritmusa bár
        nagyon hatékony, nem adhat 100\%-os védelmet az ellen, hogy egy halmazba nem tartozó elem
        átjusson a "szűrőn". A hibás átjutás valószínűségére az alábbi képlet áll rendelkezésünkre:

        \[
            (1 - e^{\frac{-km}{n}}) ^ k
        \]

        Itt a hash függvények számát k paramétert kell meghatároznunk fix \( n, m \) paraméterek esetén,
        hogy a képlet a lehető legkisebb értéket vegye fel. Ez a feladat közel sem triviális,
        egyszerű egyenlet megoldó módszerek nem juttatnak eredményre. A többváltozós analízis eszközeivel azonban meghatározható,
        hogy a válasz \( ln2 \frac{m}{n}\) . Ez a számolás hatalmas jelentőséggel bír, hiszen bármely helyzetben kihozhatjuk az algoritmusból a legtöbbet.
        A Bloom filtert az iparban számos helyen használják. A YouTube ennek segítségével állítja össze az “ajánlott” videók,
        ugyanis csak akkor kerülhet oda, ha még nem láttuk a videót. Az URL rövidítő alkalmazások ilyenek segítségével nézik meg,
        foglalt-e már az általuk generált URL cím.
        
\end{itemize}

\end{document}