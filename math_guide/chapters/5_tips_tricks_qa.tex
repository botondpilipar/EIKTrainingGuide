\documentclass[../Main.tex]{subfiles}

\begin{document}

\subsection{Hogyan készüljünk számonkérésre?}

\begin{itemize}
    \item Sok tárgyból típusfeladatok vannak
    \begin{itemize}
        \item A gyakorlatokon átvett feladatok rendszerezése, illetve az előző években
        szerepelt feladatok átnézése segíthet a típusfeladatok kiválasztásában
        \item A típusfeladatok alapos begyakorlása segíthet, hogy a számonkérés
        alkalmával ezeket gyorsan, kevesebb hibával oldjuk meg, így több idő marad
        \item Sok esetben a nehezebb feladatok is visszavezethetőek típusfeladatokra
        \begin{itemize}
            \item Ha megértjük a típusfeladat megoldásának lépéseit, könnyebben
            rájöhetünk, hogy egyes lépéseket más feladatokban is
            felhasználhatunk
            \item Néha egy-egy nehezebb feladat valójában egy típusfeladat néhány
            csavarral. Így, ha sikerül rájönnünk hogyan oldjuk meg ezeket a
            csavarokat, a megoldás további része már könnyen mehet a
            típusfeladat alapos imeretének köszönhetően
        \end{itemize}
    \end{itemize}

    \item Egyéb feladatok gyakorlása
    \begin{itemize}
        \item A gyakorlatokon megoldott feladatok gyakorlása segíthet az elmélet
        megértésében is
        \begin{itemize}
            \item Elmélyíthetjük ismereteinket, ha a feladatot összekötjük az elmélet
            tanulásával
            \begin{itemize}
                \item Egy-egy definíció, tétel megtanulása után keresünk hozzá
                feladatokat
                \item Fordítva, egy feladathoz kigyűjthetjük, milyen definíciókat,
                tételeket kell ismerjünk a megoldáshoz
            \end{itemize}
            \item A sokféle feladat megoldása segíthet, hogy a számonkérésen
            gyorsabban találjunk rá a megoldásra, hiszen többféle mintát
            ismerünk
            \begin{itemize}
                \item Ehhez beszerezhetünk többféle feladatgyűjteményt. Hasznos,
                ha megoldások (főleg ha kidolgozott, nem csak végeredmény)
                is vannak, az ellenőrzéshez a gyűjteményben
                \begin{enumerate}
                    \item Tanárok honlapjairól
                    \item Könyvtárból
                    \item Jegyzetboltokból
                \end{enumerate}
            \end{itemize}
        \end{itemize}
    \end{itemize}
    \item Az idő
    \begin{itemize}
        \item Sokat segítünk magunknak, ha időben kezdünk készülni, beosztjuk az
        időnket
        \begin{itemize}
            \item Ha több időnk van, kevesebbet kell egyszerre megtanulnunk.
            \item Ha több időnk van, és frissen tartjuk tudásunk, sokkal jobban
        \end{itemize}
        \item Sok-sok éves tapasztalatok igazolják, a zh-kra nem elég, ha az azt megelőző
        este kezdjük meg a felkészítést
        \item Egy matekos vizsgára való felkészüléshez sokszor legalább egy teljes hét
        szükséges
        \begin{itemize}
            \item Könnyebben végezhetünk tehát több vizsgával, ha már a szorgalmi
            időszakban megkezdjük a felkészülést
            \item Ha nem kell tologatjuk, halasztgatjuk a vizsgát, akkor nem
            kockáztatjuk, hogy más tárgyak annyira kicsússzanak, hogy már nem
            marad rá idő a vizsgaidőszakban
        \end{itemize}
    \end{itemize}
    \item ZH, Vizsga
    \begin{itemize}
        \item Érdemes átgondolni, hogyan osztjuk be az időnket a feladatokra
        \item Segít a tanárnak, és megelőzhetünk vitás helyzeteket, ha igyekszünk szépen
        írni
        \item Feladatok
        \begin{itemize}
            \item Ha elakadunk egy feladatban
            \begin{itemize}
                \item érdemes átgondolnunk, hasonlít-e valamelyik begyakorolt típusfeladathoz, \newline esetleg visszavezethető-e rá?
                \item hasznos lehet átgondolni, lehet-e másik megoldási módszer,
                mint amilyenekkel próbálkoztunk? Akár valami régebben, vagy
                más tárgyból tanult?
                \item vigyázzunk, nehogy túl sok időt töltsünk egy feladattal, és ne
                maradjon a többire
            \end{itemize}
            \item A tanár arra ad pontot, amit leírtunk, így érdemes mindent leírni, amit
            tudunk, illetve figyeljünk, hogy a rész-számítások is szerepeljenek a
            beadott dolgozaton
            \item Ha végeztünk egy feladattal
            \begin{itemize}
                \item Érdemes a szöveget elolvasni újra, ellenőrizve hogy tényleg
                arra válaszoltunk-e, ami a kérdés volt
                \item A számításaink ellenőrzésével elkerülhetünk
                figyelmetlenségből adódó számítási hibák miatti
                pontlevonásokat
            \end{itemize}
            \item Ha végeztünk az összes feladattal
            \begin{itemize}
                \item Ugorjunk vissza a be nem fejezett feladatokhoz
                \item Ellenőrizzük le a feladatokat
                \item Ellenőrizzük le még egyszer
            \end{itemize}
        \end{itemize}
    \end{itemize}
\end{itemize}

\end{document}