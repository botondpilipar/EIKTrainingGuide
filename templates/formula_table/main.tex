\documentclass[11pt]{article}

\usepackage[margin = 1in]{geometry}
\usepackage{amsmath}
\usepackage{amssymb}

\usepackage{multirow}


\title{Formulas}
\author{CIS 160 Fall 2014}
\date{19 September 2014}

\begin{document}
\maketitle

\setlength{\parindent}{0pt}

Here are the formulas for counting in various ways:

\begin{center}
\begin{tabular}{ |l|c|c| }
\hline
 & No Repetition & Repetition Allowed \\ \hline
\multirow{3}{*}{Not Sensitive to Order}
 & & \\
 & $ \binom{n}{r} $ & $ \left(\!\!{n\choose r}\!\!\right) = \binom{n+r-1}{r} $ \\
  & & \\

 \hline
\multirow{3}{*}{Sensitive to Order}
 & & \\
 & $ P(n,r) $ & $ n^r $ \\
  & & \\
 \hline
\end{tabular}
\end{center}

Here are examples to demonstrate:

\begin{center}
\begin{tabular}{ |l|c|c| }
\hline
 & No Repetition & Repetition Allowed \\ \hline
\multirow{3}{*}{Not Sensitive to Order}
 & {5 distinct books} & {unlimited copies of 5 books} \\
 & {choose 3 books to take home} & {choose 3 copies to take home} \\
 & & \\
 & $ \binom{5}{3} $ & $ \left(\!\!{5\choose 3}\!\!\right) = \binom{5+3-1}{3} $ \\ 
 \hline
\multirow{3}{*}{Sensitive to Order}
 & {5 distinct books} & {unlimited copies of 5 books} \\
 & {give one to person A, B, and C} & {give one to person A, B, and C} \\
 & & \\
 & $ P(5,3) $ & $ 5^3 $ \\ 
 \hline
\end{tabular}
\end{center}

And here are some more formulas: \\
\[ \binom{n}{r} = \frac{n!}{(n-r)!r!} \] \\
\[ \binom{n}{r} = \binom{n}{n-r} \] \\
\[ \binom{n}{r} = \binom{n-1}{r-1} + \binom{n-1}{r} \]



\end{document}

