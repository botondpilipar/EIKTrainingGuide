\documentclass[../Main.tex]{subfiles}
\subsection{Személyes bemutatkozás}
    • Mivel a tréningen alapból körben ülnek, érdemes csatlakozni hozzájuk a bemutatkozás erejéig, majd utána megkérni őket, hogy üljenek félkörbe, mivel a tréning tábla előtt fog zajlani.
    • Itt mindenki szabadságot kap, de szeretnénk, ha közvetlen, pozitív hangulatú és építő jellegű lenne. Ha bármikor párbeszéd alakul ki a hallgatók, vagy a hallgatók és a tréner között, azt hagyjuk kibontakozni, csak akkor rekesszük be, ha értelmetlennek találjuk.
    • Mit érdemes elmondania a trénernek a bemutatkozás során?
        ◦ Név, szak, ha programtervező informatikus, akkor szakirány, ha tanáris akkor a két tárgy.
        ◦ Miért, mióta vagy az egyetemen, mi a célod a diplomával? Bsc. után Msc., vagy akár doktori, esetleg egyből mennél dolgozni? 
        ◦ Megkérdezhetjük, ők miért jöttek az egyetemre, miért szeretnének az informatikával foglalkozni. Ki mivel szeretne foglalkozni? Játékfejlesztés, mesterséges intelligencia...?
        ◦ Dolgoztál-e már a szakmában, vannak-e tapasztalataid, amit érdemes lenne megosztani?
        ◦ Megkérdezhetjük, van-e álomcégük, ahol szeretnének dolgozni. Esetleg saját céget szeretnének-e alapítani?
        ◦ Mi az, ami szerinted első félévben fontos? (tanácsok a gólyáknak, pl. fontos a Matematikai alapok.
    • Érdemes mindenkinek a bemutatkozást előre átgondolni.
    • A Matematikai alapok fontosságára érdemes lenne itt 2 mondatban felhívni a figyelmet. Pl.: erre épül minden matek, hamar le lehet maradni, amit aztán nehéz pótolni.
\subsection{EIK és a programozás tréning bemutatása, célja}
    • A tanulásmódszertan tréning általános, de mi konkrét programozási és matematikai tréninggel készültünk: ezen az alkalmon programozás tréning lesz, a szeptember végén (szeptember 21 és 28) pedig matek.
    • A programozás tréningről: a tanulásmódszertan kurzus része, 90 perces, szünet nélkül tartjuk. A tréning három fő része:
        ◦ Programozási alapok: Algoritmizálás, kódolás.
        ◦ Az egyetemi programozás oktatás bevezetése: Miben más, miben több az ELTE IK-n a programozás, mint például egy középiskolában, vagy OKJ-s tanfolyamon? (papíron programozás, dokumentáció, specifikáció…)
        ◦ Jó programozó bemutatása: Mire kell törekedni az egyetemi anyag elsajátítása mellett? (clean code, soft skillek…)
        ◦ A tréning tartalmának vázlatát érdemes felírni a táblára
    • A tréning célja: az elsőévesek programozási ismereteinek megalapozásával segítjük a Programozás (korábbi nevén: Programozási alapismeretek) tárgy elvégzését, majd a tréning végén olyan ismereteket vezetünk be, melyek segítségükre lehetnek mind az egyetemi pályafutásuk alatt, mind a karrierjükben.
    3. EIK bemutatása
    • Pár mondatban ismertetjük az EIK-et, hogy a gólyák be tudják határolni, kik vagyunk…
        ◦ Az EIK az Eötvös Informatikai Kör rövidítése.
        ◦ Az ELTE IK PTI és tanáris hallgatói.
        ◦ A létrejöttének célja az, hogy segítse a kar hallgatóit.
        ◦ Saját tapasztalatok, élmények.