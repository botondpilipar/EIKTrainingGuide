\documentclass[../Main.tex]{subfiles}

\subsection{Személyes bemutatkozás}
Mivel a tréningen alapból körben ülnek, érdemes csatlakozni hozzájuk a bemutatkozás erejéig,
majd utána megkérni őket, hogy üljenek félkörbe, mivel a tréning tábla előtt fog zajlani.

Itt mindenki szabadságot kap, de szeretnénk, ha közvetlen, pozitív hangulatú és építő jellegű
lenne. Ha bármikor párbeszéd alakul ki a hallgatók, vagy a hallgatók és a tréner között,
azt hagyjuk kibontakozni, csak akkor rekesszük be, ha értelmetlennek találjuk.

\subsubsection{Mit érdemes elmondania a trénernek a bemutatkozás során?}

\begin{itemize}
    \item Név, szak, ha programtervező informatikus, akkor szakirány,
    ha tanáris akkor a két tárgy.
    \item Miért, mióta vagy az egyetemen, mi a célod a diplomával?
    Bsc. után Msc., vagy akár doktori, esetleg egyből mennél dolgozni?
    \item Megkérdezhetjük, ők miért jöttek az egyetemre, miért szeretnének
    az informatikával foglalkozni. Ki mivel szeretne foglalkozni?
    Játékfejlesztés, mesterséges intelligencia...?
    \item Dolgoztál-e már a szakmában, vannak-e tapasztalataid, amit érdemes lenne megosztani?
    \item Megkérdezhetjük, van-e álomcégük, ahol szeretnének dolgozni.
    Esetleg saját céget szeretnének-e alapítani?
    \item Mi az, ami szerinted első félévben fontos? 
    \textit{(tanácsok a gólyáknak, pl. fontos a Matematikai alapok.)}
\end{itemize}

\textit{\textbf{Tipp:} Érdemes mindenkinek a bemutatkozást előre átgondolni.}

\textit{\textbf{Tipp:} A Matematikai alapok fontosságára érdemes lenne itt 2 mondatban 
felhívni a figyelmet. Például erre épül minden matek, hamar le lehet maradni, amit aztán nehéz pótolni.}

\subsection{Programozás tréning bemutatása, célja}

A tanulásmódszertan tréning általános, de mi konkrét programozási és matematikai tréninggel készültünk:
ezen az alkalmon programozás tréning lesz, a szeptember végén (szeptember 21 és 28) pedig matek.

A programozás tréningről: a tanulásmódszertan kurzus része, 90 perces, szünet nélkül tartjuk.
A tréning három fő része:
\begin{enumerate}
    \item Programozási alapok: Algoritmizálás, kódolás.
    \item Az egyetemi programozás oktatás bevezetése: Miben más, miben több az ELTE IK-n a programozás,
    mint például egy középiskolában, vagy OKJ-s tanfolyamon? 
    \textit{(papíron programozás, dokumentáció, specifikáció…)}
    \item Jó programozó bemutatása: Mire kell törekedni az egyetemi anyag elsajátítása mellett?
    \textit{(clean code, soft skillek…)}
\end{enumerate}

\textit{\textbf{Tipp:} A tréning tartalmának vázlatát érdemes felírni a táblára}
        ◦ 
A tréning célja: az elsőévesek programozási ismereteinek megalapozásával segítjük a Programozás
(korábbi nevén: Programozási alapismeretek) tárgy elvégzését, majd a tréning végén olyan ismereteket
vezetünk be, melyek segítségükre lehetnek mind az egyetemi pályafutásuk alatt, mind a karrierjükben.

\subsubsection{EIK bemutatása}
Pár mondatban ismertetjük az EIK-et, hogy a gólyák be tudják határolni, kik vagyunk.

\begin{itemize}
    \item Az EIK az Eötvös Informatikai Kör rövidítése
    \item Tagjai az ELTE IK PTI és tanáris hallgatói
    \item A csoport létrejöttének célja az, hogy segítse a kar hallgatóit
\end{itemize}
\textit{\textbf{Tipp: } Ha vannak, mondjunk csoporthoz fűződő saját tapasztalatokat, élményeket.}