\documentclass[../Main.tex]{subfiles}

\begin{center}
	\begin{tabular}{| m{1.3 em} | m{0.7\textwidth} | m{0.1\textwidth} |}
		\hline
		1. & Személyes köszönet, búcsú & 3 perc \\
		\hline
		2. & EIK elérhetőségek & 1 perc \\
		\hline
		3. & Tárgyak és szakirányok említése & 1 perc \\
		\hline
		- & A \textbf{feladat teljes időtartama} & 1 perc \\
		\hline
	\end{tabular}
\end{center}

Ezzel a tréning végére értünk, reméljük, hasznotokra vált. 
Ha kérdésetek van a tréninggel, vagy bármi mással kapcsolatban, nyugodtan
keressetek meg minket személyesen, vagy írjatok az EIK csoport e-mail címére: 

\href{mailto://eik@inf.elte.hu}{Email-cím: eik@inf.elte.hu}

$\textbf{Tegyünk EIK szórólapot a tanári asztalra!}$

Találkozni fogunk még egyszer, az Egyetemi Alapozó és Tanulásmódszertan kurzus 2. szakaszában - szeptember 20. és október 4. - amikor matek tréninget fogunk tartani.

\textit{Természetesen, ha ott helyben kérdeznek, vagy szeretnének valamiről beszélni, akkor ne hagyjuk ott őket, de az idősávból ne csússzunk ki!
A tárgyakkal kapcsolatos kérdésekhez hivatalos információkat a mellékletek között lehet megtalálni }


\large{\textbf{Elsőéves tárgyak:}}

\begin{multicols}{2}

\begin{itemize}
\item Számítógépes rendszerek
\item Programozás
\item Imperatív programozás
\item Funkcionális programozás
\item Matematikai alapok
\item Innovatív vállalkozás management - nem kötelező az első félévben
\item Jogi ismeretek - nem kötelező az első félévben
\end{itemize}

\end{multicols}

\large{\textbf{Szakirányok:}}

\begin{itemize}
\item Modellező (A)
\item Szoftvertervező (B)
\item Szoftverfejlesztő (C)
\end{itemize}